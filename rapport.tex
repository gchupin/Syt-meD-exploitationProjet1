\documentclass[10pt, a4paper]{article}

\usepackage [T1]{fontenc}
\usepackage [utf8]{inputenc}
\usepackage [francais]{babel}

\title {Rapport TD 1 - Système d'exploitation}
\author {CHUPIN Guillaume & PILLEUX Julien}

\begin {document}
\maketitle
\thispagestyle{empty}
\newpage
\tableofcontents
\newpage

\section {Bilan}

\section {Points délicats}

\section {Limitations}
Les commandes ./nachos -c et ./nachos -sc ne fonctionne plus (à cause de stdinbusy qui est à 1).
Les fonctions GetInt et PutInt ne gère que les entiers entre -2147683648 à 2147483647, de part la taille du type int. De plus, pour GetInt, si à la première position on n'a pas un chiffre (excepté + et -, mais si après ce n'est toujours pas un nombre ça revient au même) alors le chiffre récupéré n'est pas le bon (dans notre cas, c'était toujours 412). \\%pour faire un retour a la ligne
Autre limitation, toujours dans ces fonctions et que si l'entier dépasse la taille de MAX\_STRING\_SIZE l'entier ne s'affiche pas complètement, il ne s'affiche que jusqu'à MAX\_STRING\_SIZE - 1 car sscanf utilise le dernier emplacement pour mettre {\tt\char`\\}0 (par exemple si on met MAX\_STRING\_SIZE à 4 et qu'on veut afficher 1234, la console n'affichera que 123).\\
Les fonctions GetChar, PutChar, GetString et PutString ne fonctionnent qu'avec les caractères ASCII.\\
Pour la fonction GetString si la chaîne de caractère dépasse la taille ``size'' du buffer, le buffer ne récupérera que les ``size'' premiers caractères. Mais, pour la fonction PutString, si on dépasse la taille du buffer ce n'est pas un problème, car il est réutilisé pour pouvoir afficher toute la chaine de caractère.  
\section {Tests}
Nous avons fait des programmes de tests pour chacun des appels systèmes crées.
\subsection {PutChar}
On a gardé le fichier de test du devoir sans le modifier outre mesure.
\subsection {GetChar}
On a essayé différents caractères sur la console et même de mettre des mots pour voir le comportement de GetChar. Les caractères récupéré par GetInt sont ensuite affiché, grâce à PutInt, pour pouvoir vérifier que l'on a récupérer le bon caractère.
\subsection {PutString}
Pour la fonction PutString on a modifier la taille de MAX\_STRING\_SIZE à 4 pour tester la fonction avec une chaîne de caractère plus grande que la taille du buffer (que l'on alloue avec MAX\_STRING\_SIZE), et ainsi remarquer si on arrive à gérer les longues chaîne de caractères.
\subsection {GetString}
On a testé avec différentes chaînes de caractères dont une plus grande que la taille du buffer qui récupère la chaîne de caractères. Ensuite on affiche cette chaîne de caractères, grâce à PutString, pour pouvoir vérifier que la chaîne de caractères récupérée est la bonne.
\subsection {PutInt}
On a fait un test en utilisant PutInt avec un nombre et son contraire.
\subsection {GetInt}
On a tester la fonction avec plein d'entrées:
\begin {itemize}
\item Des nombres positifs et négatifs plus ou moins long.
\item Des nombres mélangé à d'autres caractères.
\item Juste des caractères.
\end {itemize}
\end {document}
