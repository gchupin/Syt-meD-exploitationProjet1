\documentclass[10pt, a4paper]{article}

\usepackage [T1]{fontenc}
\usepackage [utf8]{inputenc}
\usepackage [francais]{babel}

\title {Rapport TD 1 - Système d'exploitation}
\author {CHUPIN Guillaume & PILLEUX Julien}

\begin {document}
\maketitle

\section {Bilan}

\section {Points délicats}

\section {Limitations}
Les fonctions GetInt et PutInt ne gére que les entiers entre -2147683648 à 2147483647 de part la taille du type int.\\%pour faire un retour a la ligne
Autre limitation, toujours dans ces fonctions et que si l'entier dépasse la taille de MAX\_STRING\_SIZE l'entier ne s'affiche pas complétement, il ne s'affiche que jusqu'a MAX\_STRING\_SIZE - 1 car sscanf utilise le dernier emplacement pour mettre {\tt\char`\\}0 (par exemple si on met MAX\_STRING\_SIZE à 4 et qu'on veut afficher 1234, la console n'affichera que 123).\\
Les fonctions GetChar et PutChar ne fonctionne qu'avec les caractères ASCII.\\
Pour la fonction GetString si la chaine de caractère dépasse la taille n du buffer, le buffer ne récuperrera que les n premiers caractères. 
\section {Tests}
Toutes les fonctions sont testée grâce a des fichiers de test, dans le dossier code/test, et sont appelée grace a la commande ./nachos -x ../[nom-de-la-fonction] (par exemple : ./nachos -x ../putstring pour tester la fonction PutString)
\subsection {PutChar}
On a gardé le fichier de test du devoir sans le modifier outre mesure.
\subsection {GetChar}
On a essayé différents caractères sur la console et même de mettre des mots pour voir le comportement de GetChar. Les caractères récupéré par GetInt sont ensuite affiché, grâce à PutInt, pour pouvoir vérifier que l'on a récupérer le bon caractère.
\subsection {PutString}
Pour la fonction PutString on a modifier la taille de MAX\_STRING\_SIZE à 4 pour tester la fonction avec une chaine de charactère plus grande que la taille du buffer (que l'on alloue avec MAX\_STRING\_SIZE), et ainsi remarquer si on arrive à gérer les longues chaine de charactères.
\subsection {GetString}
On a testé avec différentes chaines de caractères dont une plus grande que la taille du buffer qui récupère la chaine de caractères. Ensuite on affiche cette chaine de caractères, grâce à PutString, pour pouvoir vérifier que la chaine de caractères récupérée est la bonne.
\end {document}
